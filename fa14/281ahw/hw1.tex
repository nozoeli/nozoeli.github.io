%%%%%%%%%%%%%%%%%%%%%%%%%%%%%%%%%%%%%%%%%
% Short Sectioned Assignment
% LaTeX Template
% Version 1.0 (5/5/12)
%
% This template has been downloaded from:
% http://www.LaTeXTemplates.com
%
% Original author:
% Frits Wenneker (http://www.howtotex.com)
%
% License:
% CC BY-NC-SA 3.0 (http://creativecommons.org/licenses/by-nc-sa/3.0/)
%
%%%%%%%%%%%%%%%%%%%%%%%%%%%%%%%%%%%%%%%%%

%----------------------------------------------------------------------------------------
%	PACKAGES AND OTHER DOCUMENT CONFIGURATIONS
%----------------------------------------------------------------------------------------

\documentclass[paper=a4, fontsize=11pt]{scrartcl} % A4 paper and 11pt font size

\usepackage[T1]{fontenc} % Use 8-bit encoding that has 256 glyphs
\usepackage{fourier} % Use the Adobe Utopia font for the document - comment this line to return to the LaTeX default
\usepackage[english]{babel} % English language/hyphenation
\usepackage{amsmath,amsfonts,amsthm} % Math packages

\usepackage{lipsum} % Used for inserting dummy 'Lorem ipsum' text into the template

\usepackage{sectsty} % Allows customizing section commands
\allsectionsfont{\centering \normalfont\scshape} % Make all sections centered, the default font and small caps

\usepackage{fancyhdr} % Custom headers and footers
\pagestyle{fancyplain} % Makes all pages in the document conform to the custom headers and footers
\fancyhead{} % No page header - if you want one, create it in the same way as the footers below
\fancyfoot[L]{} % Empty left footer
\fancyfoot[C]{} % Empty center footer
\fancyfoot[R]{\thepage} % Page numbering for right footer
\renewcommand{\headrulewidth}{0pt} % Remove header underlines
\renewcommand{\footrulewidth}{0pt} % Remove footer underlines
\setlength{\headheight}{13.6pt} % Customize the height of the header
\newenvironment{theorem}[2][Theorem]{\begin{trivlist}
\item[\hskip \labelsep {\bfseries #1}\hskip \labelsep {\bfseries #2.}]}{\end{trivlist}}
\newenvironment{lemma}[2][Lemma]{\begin{trivlist}
\item[\hskip \labelsep {\bfseries #1}\hskip \labelsep {\bfseries #2.}]}{\end{trivlist}}
\newenvironment{exercise}[2][Exercise]{\begin{trivlist}
\item[\hskip \labelsep {\bfseries #1}\hskip \labelsep {\bfseries #2.}]}{\end{trivlist}}
\newenvironment{problem}[2][Problem]{\begin{trivlist}
\item[\hskip \labelsep {\bfseries #1}\hskip \labelsep {\bfseries #2.}]}{\end{trivlist}}
\newenvironment{question}[2][Question]{\begin{trivlist}
\item[\hskip \labelsep {\bfseries #1}\hskip \labelsep {\bfseries #2.}]}{\end{trivlist}}
\newenvironment{corollary}[2][Corollary]{\begin{trivlist}
\item[\hskip \labelsep {\bfseries #1}\hskip \labelsep {\bfseries #2.}]}{\end{trivlist}}
\numberwithin{equation}{section} % Number equations within sections (i.e. 1.1, 1.2, 2.1, 2.2 instead of 1, 2, 3, 4)
\numberwithin{figure}{section} % Number figures within sections (i.e. 1.1, 1.2, 2.1, 2.2 instead of 1, 2, 3, 4)
\numberwithin{table}{section} % Number tables within sections (i.e. 1.1, 1.2, 2.1, 2.2 instead of 1, 2, 3, 4)
\newcommand{\E}{\operatorname{\mathbb{E}}}
\renewcommand{\P}{\operatorname{\mathbb{P}}}
\newcommand{\Var}{\operatorname{Var}}
\newcommand{\Cov}{\operatorname{Cov}}
\newcommand{\Cor}{\operatorname{Cor}}
\newcommand{\expect}[1]{\mathbb{E}\left(#1\right)}
\newcommand{\pr}[1]{\mathbb{P}\left(#1\right)}
\newcommand{\var}[1]{\operatorname{Var}\left(#1\right)}
\newcommand{\cov}[1]{\operatorname{Cov}\left(#1\right)}
%\newcommand{\cor}[1]{\operatorname{Cor}\left(#1\right)}
\newcommand\indep{\protect\mathpalette{\protect\independenT}{\perp}}
\def\independenT#1#2{\mathrel{\rlap{$#1#2$}\mkern2mu{#1#2}}}

\def\iid{\stackrel{\rm iid}{\sim}}
\def\Bin{\text{Bin}}
\def\Unif{\text{Unif}}
\def\lsto{\stackrel{\rm sto}{\leq}}
\def\gsto{\stackrel{\rm sto}{\geq}}

\setlength\parindent{0pt} % Removes all indentation from paragraphs - comment this line for an assignment with lots of text

%----------------------------------------------------------------------------------------
%	TITLE SECTION
%----------------------------------------------------------------------------------------

\newcommand{\horrule}[1]{\rule{\linewidth}{#1}} % Create horizontal rule command with 1 argument of height

\title{	
\normalfont \normalsize 
\textsc{UC San Diego, Department of Mathematics} \\ [25pt] % Your university, school and/or department name(s)
\horrule{0.5pt} \\[0.4cm] % Thin top horizontal rule
\huge MATH281A Homework 1 \\ % The assignment title
\horrule{2pt} \\[0.5cm] % Thick bottom horizontal rule
}

\date{\normalsize\today} % Today's date or a custom date

\begin{document}

\maketitle



\begin{exercise}{1.1.2} 
\end{exercise}

Hint: Use Lagrangian multipliers. The basic idea is that since it is iid then everything should be symmetry. 

\begin{exercise}{1.1.3} 
\end{exercise}

Pay attention the $\alpha_i$s in (a) are fixed. So they are not the variables you are going to optimize. Lagrangian multipliers will work, but if you define $Y_i = \sqrt{\alpha_i} X_i$, things will be better.

For (b), just use
$$
var (\sum_i \alpha_i X_i) = \sum_i \alpha^2_i var(X_i) + \sum_{i \neq j} \alpha_i \alpha_j cov(X_i,X_j)
$$

\begin{exercise}{1.1.12} 
\end{exercise}

Go direct integration and the answer will come out. To show the finite moment you may need to control the tail with inequality
$$
\frac{1}{(1+|x|)^k} < \frac{1}{|x|^k}
$$

\begin{exercise}{1.4.1} 
\end{exercise}

Hint: Use the cdf and do a transformation then you will find what you need.

\begin{exercise}{1.4.2} 
\end{exercise}

Hint: Cdf still works. For instance:
$$
\P (-\log X < x) = \P ( X < e^{-x}) 
$$
Then compare the result and the cdf of an exponentially distributed variable.

\begin{exercise}{1.4.13b} 
\end{exercise}

The same with the previous questions. Check Weibull distribution and its cdf in Wikipedia.

\begin{exercise}{class.1} 
\end{exercise}

Hint: Use equality
$$
\sum_i (X_i - \bar{X})^2 = \sum_i X_i^2 - n(\bar{X})^2
$$
and note that $\E (X^2_i) = var (X_i) + (\E(X))^2$. Do the same trick to find $\E(\bar{X})^2$.

\begin{exercise}{class.2} 
\end{exercise}

This equation may make life easier:
$$
\sum_i (X_i - \bar{X})^2 = \sum_{\{i,j\}} \frac{1}{n} (X_i - X_j)^2
$$
The right hand side runs all combination of $\{i,j\}$. First prove the equation above, then carefully analyze your target using the formula for variances.

\begin{exercise}{class.3} 
\end{exercise}

For skewness, try to prove that for iid sample $X_1,\ldots , X_n$, we have
$$
\mu_3 (\sum_i X_i) = n \mu_3(X_i)
$$
Here $\mu_3$ stands for third central moment.

For kurtosis, try to show that for iid sample $X_1,\ldots , X_n$, we have
$$
\mu_4 (\sum_i X_i) - 3(\mu_2(\sum_i X_i))^2 = n (\mu_4(X_i) - 3\mu_2(X_i))
$$
Here $\mu_4$ is 4th central moment and $\mu_2$ is 2nd central moment (variance).
\end{document}
