\documentclass[12pt]{article}
\usepackage{amssymb}
\usepackage{amsfonts}
\usepackage{amsfonts}
\usepackage{amssymb}
\usepackage[margin=1in]{geometry}
\usepackage{amsmath,amsthm,amssymb}
\newcommand{\N}{\mathbb{N}}
\newcommand{\Z}{\mathbb{Z}}
\newenvironment{theorem}[2][Theorem]{\begin{trivlist}
\item[\hskip \labelsep {\bfseries #1}\hskip \labelsep {\bfseries #2.}]}{\end{trivlist}}
\newenvironment{lemma}[2][Lemma]{\begin{trivlist}
\item[\hskip \labelsep {\bfseries #1}\hskip \labelsep {\bfseries #2.}]}{\end{trivlist}}
\newenvironment{exercise}[2][Exercise]{\begin{trivlist}
\item[\hskip \labelsep {\bfseries #1}\hskip \labelsep {\bfseries #2.}]}{\end{trivlist}}
\newenvironment{problem}[2][Problem]{\begin{trivlist}
\item[\hskip \labelsep {\bfseries #1}\hskip \labelsep {\bfseries #2.}]}{\end{trivlist}}
\newenvironment{question}[2][Question]{\begin{trivlist}
\item[\hskip \labelsep {\bfseries #1}\hskip \labelsep {\bfseries #2.}]}{\end{trivlist}}
\newenvironment{corollary}[2][Corollary]{\begin{trivlist}
\item[\hskip \labelsep {\bfseries #1}\hskip \labelsep {\bfseries #2.}]}{\end{trivlist}}
\newcommand{\E}{\operatorname{\mathbb{E}}}
\renewcommand{\P}{\operatorname{\mathbb{P}}}
\newcommand{\Var}{\operatorname{Var}}
\newcommand{\Cov}{\operatorname{Cov}}
\newcommand{\Cor}{\operatorname{Cor}}
\newcommand{\expect}[1]{\mathbb{E}\left(#1\right)}
\newcommand{\pr}[1]{\mathbb{P}\left(#1\right)}
\newcommand{\var}[1]{\operatorname{Var}\left(#1\right)}
\newcommand{\cov}[1]{\operatorname{Cov}\left(#1\right)}
%\newcommand{\cor}[1]{\operatorname{Cor}\left(#1\right)}
\newcommand\indep{\protect\mathpalette{\protect\independenT}{\perp}}
\def\independenT#1#2{\mathrel{\rlap{$#1#2$}\mkern2mu{#1#2}}}

\def\iid{\stackrel{\rm iid}{\sim}}
\def\Bin{\text{Bin}}
\def\Unif{\text{Unif}}
\def\lsto{\stackrel{\rm sto}{\leq}}
\def\gsto{\stackrel{\rm sto}{\geq}}

\begin{document}
% --------------------------------------------------------------
% Start here
% --------------------------------------------------------------
\title{Homework 6}%replace X with the appropriate number
\author{MATH 281A} %if necessary, replace with your course title
\maketitle
\begin{exercise}{3.18}
\end{exercise}

We use Method I to derive the contradiction. Pay special attention to the constant coefficient in both side of the polynomials. 

\begin{exercise}{3.19}
\end{exercise}

This is a tricky one. Pay attention that the summation of the two truncated Poisson variables is NOT a truncated Poisson with parameter as the summation of the two parameters.

To attack the problem, go through these steps. We set $n=1$. First calculate the mean of the truncated Poisson. One easy way to do this is using
$$
\theta = (1 - \P (\tilde{X} = 0) ) \E (X)
$$
where $X$ follows truncated Poisson and $\tilde{X}$ follows Poisson with the same distribution $\theta$. (Why we have this equation? Try to derive it.)

You may find that this equation does not really solve the question since $\E(X)$ is not a linear function of $\theta$. One more ingredient is to consider $\P(X = 1)$ by using the following
$$
\P(X = 1) = \frac{\P(\tilde{X} = 1)}{1 - \P(\tilde{X} = 0)}
$$
where the setting is the same as before. 

Finally try to construct $\theta$ using $\E(X)$ and $\P(X =1)$. What is the statistic finally?

For $n =2$, by a simple exponential family observation we can find sufficient and complete statistic as $X_1 + X_2$. Then suppose the UMVUE in part (a) is $T(X_1)$, we just need
$$
\E (T(X_1) | X_1 +X_2)
$$

You will encounter several problems when doing this Rao-Blackwellization stuff. If you get stuck, try augmenting the variable into full Poisson, then get the result and deduct the case when zero happens.

Good luck...

\begin{exercise}{20(C)}
\end{exercise}

Direct calculation may yield the desired result. You only need to verify the C-R bound.

\begin{exercise}{2.1}
\end{exercise}

Follow the hint and you will be OK. Notice that $\sigma^2$ is a constant.

\begin{exercise}{2.5}
\end{exercise}

Use the standard move to show that the statistic is unbiased and based on the sufficient and complete statistic.

\begin{exercise}{2.9}
\end{exercise}

All you need to do is to show $X$ sufficient and complete...

\begin{exercise}{2.15}
\end{exercise}

We may have done this question before.

\begin{exercise}{2.25}
\end{exercise}

This one is also not easy. First exhibit that $(X_{(n)}, Y_{(n)})$ is complete and sufficient (Easy). And intuitively we may look at statistic $X_{(n)} / Y_{(n)}$. To calculate the expectation of the statistic, the following lemma may be helpful.

For some nonnegative continuous random variable $X$, we have
$$
\E(X) = \int_0^\infty \P(X > t) dt
$$

\begin{exercise}{2.27}
\end{exercise}

Once we have (a), then (b) is obvious since $MSE = bias^2 + Variance$ and the variance of MLE is clearly smaller  by the hint.

For (a), note that $\xi = u$ is equivalent to the fact that $p = \Phi(\xi - u) = 0.5$. Then we need to show that $\E \Phi (u - \bar{X}) = 0.5$. In fact you just need to show that $\Phi (u - \bar{X})$ follows uniform distribution on $(0,1)$. 

% --------------------------------------------------------------
% You don't have to mess with anything below this line.
% --------------------------------------------------------------
\end{document} 