\documentclass[12pt]{article}
\usepackage{amssymb}
\usepackage{amsfonts}
\usepackage{amsfonts}
\usepackage{amssymb}
\usepackage{bbm}
\usepackage[margin=1in]{geometry}
\usepackage{amsmath,amsthm,amssymb}
\newcommand{\N}{\mathbb{N}}
\newcommand{\Z}{\mathbb{Z}}
\newcommand{\1}{\mathbbm{1}}
\newenvironment{theorem}[2][Theorem]{\begin{trivlist}
\item[\hskip \labelsep {\bfseries #1}\hskip \labelsep {\bfseries #2.}]}{\end{trivlist}}
\newenvironment{lemma}[2][Lemma]{\begin{trivlist}
\item[\hskip \labelsep {\bfseries #1}\hskip \labelsep {\bfseries #2.}]}{\end{trivlist}}
\newenvironment{exercise}[2][Exercise]{\begin{trivlist}
\item[\hskip \labelsep {\bfseries #1}\hskip \labelsep {\bfseries #2.}]}{\end{trivlist}}
\newenvironment{problem}[2][Problem]{\begin{trivlist}
\item[\hskip \labelsep {\bfseries #1}\hskip \labelsep {\bfseries #2.}]}{\end{trivlist}}
\newenvironment{question}[2][Question]{\begin{trivlist}
\item[\hskip \labelsep {\bfseries #1}\hskip \labelsep {\bfseries #2.}]}{\end{trivlist}}
\newenvironment{corollary}[2][Corollary]{\begin{trivlist}
\item[\hskip \labelsep {\bfseries #1}\hskip \labelsep {\bfseries #2.}]}{\end{trivlist}}
\newcommand{\E}{\operatorname{\mathbb{E}}}
\renewcommand{\P}{\operatorname{\mathbb{P}}}
\newcommand{\Var}{\operatorname{Var}}
\newcommand{\Cov}{\operatorname{Cov}}
\newcommand{\Cor}{\operatorname{Cor}}
\newcommand{\expect}[1]{\mathbb{E}\left(#1\right)}
\newcommand{\pr}[1]{\mathbb{P}\left(#1\right)}
\newcommand{\var}[1]{\operatorname{Var}\left(#1\right)}
\newcommand{\cov}[1]{\operatorname{Cov}\left(#1\right)}
%\newcommand{\cor}[1]{\operatorname{Cor}\left(#1\right)}
\newcommand\indep{\protect\mathpalette{\protect\independenT}{\perp}}
\def\independenT#1#2{\mathrel{\rlap{$#1#2$}\mkern2mu{#1#2}}}

\def\iid{\stackrel{\rm iid}{\sim}}
\def\Bin{\text{Bin}}
\def\Unif{\text{Unif}}
\def\lsto{\stackrel{\rm sto}{\leq}}
\def\gsto{\stackrel{\rm sto}{\geq}}

\begin{document}
% --------------------------------------------------------------
% Start here
% --------------------------------------------------------------
\title{Homework 3}%replace X with the appropriate number
\author{MATH 281B} %if necessary, replace with your course title
\maketitle
\begin{exercise}{Show ARE of sample mean and median for unimodal model $\geq 1/3$}
\end{exercise}

Without loss of generality we assume the mean and median as $0$ and the distribution is symmetry. Then we see that $f(0)$ is the largest value of the density function. The median has an asymptotic variance of $1/(4f(0)^2)$ and mean has variance $\sigma^2$. The target is to show that
$$
 \sigma^2 \geq \frac{1}{3} \cdot \frac{1}{4f(0)^2}
$$
or
$$
f(0)^2 \int x^2 f(x)dx \geq \frac{1}{12}
$$

This can be done by considering another distribution $g$ which is a uniform distribution on $[-1/(2f(0)) , 1/2(f(0))]$. First argue that $g$ has a smaller variance of $f$, and replace the target integration with $g$.

\begin{exercise}{Unstability of median}
\end{exercise}

The target is to show that the asymptotic variance of the sample mean of the distribution $f (x) = |x| $ on $[-1,1]$ is infinity. We finish this by showing the second moment, multiplied by $n$, is infinity.

First, for $n$ even, try to show the density of the sample mean is
$$
f_n(x) = C_n (\frac{1}{4} - \frac{|x|^4}{4})^{n/2} |x| dx
$$

Second, write out the expression of the second moment as
$$
n\E (|X|^2) = 2n \int_0^1  C_n (\frac{1}{4} - \frac{|x|^4}{4})^{n/2} |x|^3 dx = 2nC_n \int_0^{1/4} (\frac{1}{4} -t)^{n/2} dt = 2\frac{n}{n/2+1}C_n (1/4)^{n/2+1}
$$
To hold this finite, we must have $C_n (1/4)^{n/2}$ asymptotically finite. But then the density function will break down to zero. Try to argue this.
% --------------------------------------------------------------
% You don't have to mess with anything below this line.
% --------------------------------------------------------------
\end{document} 