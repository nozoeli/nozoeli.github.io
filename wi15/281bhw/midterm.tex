\documentclass[12pt]{article}
\usepackage{amssymb}
\usepackage{amsfonts}
\usepackage{amsfonts}
\usepackage{amssymb}
\usepackage{bbm}
\usepackage[margin=1in]{geometry}
\usepackage{amsmath,amsthm,amssymb}
\newcommand{\N}{\mathbb{N}}
\newcommand{\Z}{\mathbb{Z}}
\newcommand{\1}{\mathbbm{1}}
\newenvironment{theorem}[2][Theorem]{\begin{trivlist}
\item[\hskip \labelsep {\bfseries #1}\hskip \labelsep {\bfseries #2.}]}{\end{trivlist}}
\newenvironment{lemma}[2][Lemma]{\begin{trivlist}
\item[\hskip \labelsep {\bfseries #1}\hskip \labelsep {\bfseries #2.}]}{\end{trivlist}}
\newenvironment{exercise}[2][Exercise]{\begin{trivlist}
\item[\hskip \labelsep {\bfseries #1}\hskip \labelsep {\bfseries #2.}]}{\end{trivlist}}
\newenvironment{problem}[2][Problem]{\begin{trivlist}
\item[\hskip \labelsep {\bfseries #1}\hskip \labelsep {\bfseries #2.}]}{\end{trivlist}}
\newenvironment{question}[2][Question]{\begin{trivlist}
\item[\hskip \labelsep {\bfseries #1}\hskip \labelsep {\bfseries #2.}]}{\end{trivlist}}
\newenvironment{corollary}[2][Corollary]{\begin{trivlist}
\item[\hskip \labelsep {\bfseries #1}\hskip \labelsep {\bfseries #2.}]}{\end{trivlist}}
\newcommand{\E}{\operatorname{\mathbb{E}}}
\renewcommand{\P}{\operatorname{\mathbb{P}}}
\newcommand{\Var}{\operatorname{Var}}
\newcommand{\Cov}{\operatorname{Cov}}
\newcommand{\Cor}{\operatorname{Cor}}
\newcommand{\expect}[1]{\mathbb{E}\left(#1\right)}
\newcommand{\pr}[1]{\mathbb{P}\left(#1\right)}
\newcommand{\var}[1]{\operatorname{Var}\left(#1\right)}
\newcommand{\cov}[1]{\operatorname{Cov}\left(#1\right)}
%\newcommand{\cor}[1]{\operatorname{Cor}\left(#1\right)}
\newcommand\indep{\protect\mathpalette{\protect\independenT}{\perp}}
\def\independenT#1#2{\mathrel{\rlap{$#1#2$}\mkern2mu{#1#2}}}

\def\iid{\stackrel{\rm iid}{\sim}}
\def\Bin{\text{Bin}}
\def\Unif{\text{Unif}}
\def\lsto{\stackrel{\rm sto}{\leq}}
\def\gsto{\stackrel{\rm sto}{\geq}}

\begin{document}
% --------------------------------------------------------------
% Start here
% --------------------------------------------------------------
\title{Midterm}%replace X with the appropriate number
\author{MATH 281B} %if necessary, replace with your course title
\maketitle
\begin{problem}{1}
\end{problem}

(a). The shape parameter is $alpha$ and the scale parameter is $\beta$. The way to determine this is that if $X$ follows $\Gamma(\alpha,\beta)$ then from the density function we can tell that $cX$ is following $Gamma(\alpha,c\beta)$. Therefore $\beta$ is a scale parameter.

(b). The posterior $\pi(\lambda | X)$ is following
$$
\pi (\lambda |X) \propto \pi(\lambda) f(X|\lambda) \propto \lambda^{\alpha -1} e^{-\lambda/\beta} \lambda^{\sum_i X_i} e^{-n\lambda} \ = \lambda^{\alpha+\sum_i X_i -1 }  e^{-(n+1/\beta)\lambda}
$$
Observe and we can find that the right hand side is the kernel of the Gamma distribution with parameter $(\alpha +\sum_i X_i, \beta/(n\beta +1))$, so it is belonging to the Gamma family.

(c). From the formula of Gamma distribution, the posterior mean is 
$$
(\alpha +\sum_i X_i) \cdot \frac{\beta}{(n\beta +1)}
$$ 
The reason to choose posterior mean to be the Bayes estimator is that it minimize the Bayes risk \textbf{under the squared loss function}. A simple argument can be as follows.
$$
\E_{\lambda|X} ((\delta - \lambda)^2 ) = \E_{\lambda|X} (\delta^2 - 2\delta \lambda + \lambda^2 ) = \delta^2 -2\delta \E (\lambda|X) + \E (\lambda^2|X)
$$
Then it is trivial to see that $\delta  = \E (\lambda|X)$ minimizes the function.

(d). Direct algebra shows that
$$
\hat{\lambda} = (\alpha +\sum_i X_i) \cdot \frac{\beta}{(n\beta +1)} = \frac{1}{n\beta +1} \cdot (\alpha \beta) + \frac{n\beta}{n\beta +1} \cdot \bar{X},
$$
where $\alpha \beta$ is the mean of the prior, and $\bar{X}$ is the usual frequentist's estimator.

(e). From central limit theorem we know that
$$
\sqrt{n} (\bar{X} - \lambda) \xrightarrow{D} N(0,\lambda),
$$
since Poisson distribution has mean and variance $\lambda$. What's more we have
$$
\frac{n\beta}{n\beta +1} \rightarrow 1,  \frac{\sqrt{n}}{n\beta +1}  \rightarrow 0,
$$
therefore if we combine the parts together, along with Slutsky's lemma, we well get
$$
\sqrt{n} (\hat{\lambda} - \lambda) = \sqrt{n} (\frac{n\beta}{n\beta +1} \cdot \bar{X} - \lambda) + \frac{\sqrt{n}}{n\beta +1} \cdot (\alpha \beta) \xrightarrow{D} N(0,\lambda)
$$

(f). Since $\hat{\lambda}$ is the unique Bayes estimator under the Bayes rule, it is admissible.

(g). It is not true that the posterior variance is guaranteed smaller. However, if we do the variance decomposition,
$$
\var{\lambda}  = \var{\E (\lambda|X)} + \E (\var{ \lambda |X}) \geq \E (\var{ \lambda |X}) ,
$$
we can see that the prior variance is no less than the \textbf{expectation} of the posterior variance. Clearly this gives no clue of any guaranteed behavior, but you may claim that with $n$ increasing, the probability of getting a smaller posterior variance is increasing to $1 $.

(h). By another version of the variance decomposition, we have
$$
\var{X} = \var{ \E (X |\lambda) } + \E( \var{X|\lambda}).
$$
Note that $X|\lambda$ follows Poisson distribution indexed by $\lambda$. Therefore 
$$
\E (X |\lambda) = \var{X|\lambda} = \lambda.
$$
And clearly
$$
\var{X} = \var{ \lambda } + \E( \lambda) = \alpha \beta + \alpha \beta^2
$$
since $\lambda $ follows $\Gamma(\alpha,\beta)$.

\begin{problem}{2}
\end{problem}

(a). By the behavior of the sample median we have
$$
\sqrt{n} (\hat{\alpha}_1 - \alpha) \xrightarrow{D} N(0, \frac{1}{4f(\alpha)^2}),
$$ 
where $f(\cdot)$ is the density function of the uniform distribution. Plug in the numbers we can get 
$$
\sqrt{n} (\hat{\alpha}_1 - \alpha) \xrightarrow{D} N(0, 4).
$$ 

For $\hat{\alpha}_2$, central limit theorem tells us
$$
\sqrt{n} (\hat{\alpha}_2 - \alpha) \xrightarrow{D} N(0, \var{X_1}),
$$
and we have $\var{X_1} = (\alpha+2 - (\alpha-2) )^2 /12 = 4/3$. Therefore
$$
\sqrt{n} (\hat{\alpha}_2 - \alpha) \xrightarrow{D} N(0, \frac{4}{3}).
$$
Thus we have the ARE as
$$
ARE(\hat{\alpha}_2 , \hat{\alpha}_1 ) = \frac{4}{\frac{4}{3}} = 3
$$

(b). If $\hat{\alpha}_1$ uses $1000$ samples, then since $\hat{\alpha}_2$ has three times asymptotic efficiency, it will only need $1000/3 = 334$ samples to match the performance.

(c). The MLE, or the UMVUE, based what have been learned in 281A, is found by 
$$
\hat{\alpha}_3 = \frac{X_{(1)} + X_{(n)}}{2}.
$$

There are many ways to show that this estimator is converging to $\alpha$ with speed $n$, which is much faster than $\hat{\alpha}_1$ and $\hat{\alpha}_2$, with speed $\sqrt n$. And this tells you that $ARE(\hat{\alpha}_1,\hat{\alpha}_3) = 0$. 

To show this, we first bound the variance of $\hat{\alpha}_3$. By Cauchy-Schwartz inequality,
$$
\var{\hat{\alpha}_3} = \var{\frac{X_{(1)} + X_{(n)}}{2}} \leq 2(\var{\frac{X_{(1)}}{2}}) + 2(\var{\frac{X_{(n)}}{2}}).
$$
And, use the property of this symmetric distribution, we can expect $\var{X_{(1)}} = \var{X_{(n)}}$. Therefore 
$$
\var{\hat{\alpha}_3} \leq 4\var{\frac{X_{(n)}}{2}} = \var{X_{(n)}}
$$

By the probability equality,
$$
\P (X_{(n)} \leq x) = \P (X_i \leq x, i=1,\ldots,n ) = \P(X_1 \leq x)^n = (\frac{x - (\alpha-2)}{4})^n,
$$
we have the density for $X_{(n)}$ as
$$
f_{X_{(n)}} (x) = \frac{n(x - (\alpha-2))^{n-1}}{4^n}.
$$
By simple integration we can find that
$$
\E (X_{(n)}) = \frac{n}{n+1} = \frac{4n}{n+1} + \alpha -2
$$
and 
$$
\E (X_{(n)}^2) = \frac{16n}{n+2} + 2(\alpha -2) ( \frac{4n}{n+1} ) + (\alpha-2)^2.
$$
Then the variance is given by
$$
\var{X_{(n)}} = \E (X_{(n)}^2) - (\E (X_{(n)}))^2 = \frac{16n}{(n+1)^2(n+2)}.
$$
Thus, inflating $\hat{\alpha}_3$ by $\sqrt n$ will still give an asymptotic variance $0$, which shows that $ARE(\hat{\alpha}_1,\hat{\alpha}_3) = 0$.

Another way is recognizing that asymptotically $n( 2+\alpha - X_{(n)})$ is following exponential distribution. The argument is as follows. Denote $T =  n( 2+\alpha - X_{(n)})$, and
$$
\P (T \geq t ) = \P (X_{(n)} \leq 2+\alpha- \frac{a}{n} ) = (1 - \P (X_1 \geq 2+\alpha- \frac{a}{n} ))^n = (1 - \frac{a}{4n})^n \rightarrow e^{-\frac{a}{4}}.
$$
Therefore $T$ is asymptotically distributed as $exp(4)$. Since $\sqrt{n}(\hat{\alpha}_3)$ has a smaller variance of $T /\sqrt{n}$, and the latter has asymptotic variance going to zero, we conclude that $\sqrt{n}(\hat{\alpha}_3)$ has asymptotic variance zero, which indicates that $ARE(\hat{\alpha}_1,\hat{\alpha}_3) = 0$.

% --------------------------------------------------------------
% You don't have to mess with anything below this line.
% --------------------------------------------------------------
\end{document} 