\documentclass[12pt]{article}
\usepackage{amssymb}
\usepackage{amsfonts}
\usepackage{amsfonts}
\usepackage{amssymb}
\usepackage{bbm}
\usepackage[margin=1in]{geometry}
\usepackage{amsmath,amsthm,amssymb}
\newcommand{\N}{\mathbb{N}}
\newcommand{\Z}{\mathbb{Z}}
\newcommand{\1}{\mathbbm{1}}
\newenvironment{theorem}[2][Theorem]{\begin{trivlist}
\item[\hskip \labelsep {\bfseries #1}\hskip \labelsep {\bfseries #2.}]}{\end{trivlist}}
\newenvironment{lemma}[2][Lemma]{\begin{trivlist}
\item[\hskip \labelsep {\bfseries #1}\hskip \labelsep {\bfseries #2.}]}{\end{trivlist}}
\newenvironment{exercise}[2][Exercise]{\begin{trivlist}
\item[\hskip \labelsep {\bfseries #1}\hskip \labelsep {\bfseries #2.}]}{\end{trivlist}}
\newenvironment{problem}[2][Problem]{\begin{trivlist}
\item[\hskip \labelsep {\bfseries #1}\hskip \labelsep {\bfseries #2.}]}{\end{trivlist}}
\newenvironment{question}[2][Question]{\begin{trivlist}
\item[\hskip \labelsep {\bfseries #1}\hskip \labelsep {\bfseries #2.}]}{\end{trivlist}}
\newenvironment{corollary}[2][Corollary]{\begin{trivlist}
\item[\hskip \labelsep {\bfseries #1}\hskip \labelsep {\bfseries #2.}]}{\end{trivlist}}
\newcommand{\E}{\operatorname{\mathbb{E}}}
\renewcommand{\P}{\operatorname{\mathbb{P}}}
\newcommand{\Var}{\operatorname{Var}}
\newcommand{\Cov}{\operatorname{Cov}}
\newcommand{\Cor}{\operatorname{Cor}}
\newcommand{\expect}[1]{\mathbb{E}\left(#1\right)}
\newcommand{\pr}[1]{\mathbb{P}\left(#1\right)}
\newcommand{\var}[1]{\operatorname{Var}\left(#1\right)}
\newcommand{\cov}[1]{\operatorname{Cov}\left(#1\right)}
%\newcommand{\cor}[1]{\operatorname{Cor}\left(#1\right)}
\newcommand\indep{\protect\mathpalette{\protect\independenT}{\perp}}
\def\independenT#1#2{\mathrel{\rlap{$#1#2$}\mkern2mu{#1#2}}}

\def\iid{\stackrel{\rm iid}{\sim}}
\def\Bin{\text{Bin}}
\def\Unif{\text{Unif}}
\def\lsto{\stackrel{\rm sto}{\leq}}
\def\gsto{\stackrel{\rm sto}{\geq}}

\begin{document}
% --------------------------------------------------------------
% Start here
% --------------------------------------------------------------
\title{1. Probability Theory}%replace X with the appropriate number
\author{GULP Winter 2015} %if necessary, replace with your course title
\maketitle

\section{Random Variables}

We may think that the random variables (r.v.) are basically some random numbers. In graduate text the r.v. are \textbf{measurable functions} from the \textbf{sample space} to the real number space $\mathbb{R}^n$.

\textbf{Example}. Toss the coin twice. The sample space, which is all the result you can get, is {HH,TT,HT,TH}. Define the function as follows: $X = 1$ when you have two same result and $X = 0$ otherwise. Then $X(HT) = 0$, and $X(HH) = X(TT) = 1$.

As we define r.v. as functions, we can do integrations on it. The integration on the r.v. are done against the \textbf{probability measure}, which assigns weights to the sample space.

Like what we have in the language of analysis, if the r.v. has finite first moment, we call it is in $L^1$, or have a \textbf{finite mean}. Furthermore if it has finite second moment, we may conclude that it is in $L^2$, or has a \textbf{finite variance}. In probability space you always have finite lower moments when you have finite higher moments.

\section{Densities and the MLE}

It is not true anymore that only the continuous r.v. have the probability density. The density is used to describe the \textbf{possibility that some data appears}.

\textbf{Example}. The density of a continuous random variable is defined as
$$
f(x) = \lim_{\Delta x \rightarrow 0} \frac{F(x + \Delta x) - F(x)}{\Delta x}
$$
where $F$ is the cdf for the r.v. If we do a bit of transformation:
$$
f(x) \Delta x = F(x + \Delta x) - F(x) = \P (X \in (x, x + \Delta x))
$$
it is roughly the probability that $X$ falls into a small interval. Since the probability itself will go to zero when the interval shrinks to a number, we magnify the probability by dividing $\Delta x$ so that it becomes the density function.

You may see that the pdf represents the comparative probability so that if we shift the pdf by a constant, it does not change anything if we do MLE or inference on that. 

For discrete functions, the density function is naturally the probability mass function $\P (X = x_i)$, where $\{x_i\}$ are possible values that $X$ can take. Clearly the density represents the possibility that data happens in this case.

Now we can define the density for other random variables. 

\textbf{Example} Describe the density function for $X$ such that $\P(X = 0) =0.5$ and $\P(X = Y) = 0.5$ where $Y$ follows an exponential distribution with parameter $1$.

\textbf{Example} Describe the density function for $X$ such that $\P(X = 0) =\theta $ and $\P(X = Y) = 0.5$ where $Y$ follows an normal distribution with mean $\theta$ and variance $1$.

When we have the idea of densities we can expand the idea of MLE. 

\textbf{Example}(cont'd) Describe the likelihood function when we have $n$ samples in the previous case.
$$
L(\theta) =\prod_{i=1}^n \theta^{\1(X_i = 0)} \phi_{\theta,1}(X_i)^{\1 (X_i \neq 0)}
$$
where
$$
\phi_{\theta, 1}(x) =\frac{1}{\sqrt{2\pi}} e^{\frac{(x- \theta)^2}{2}}
$$
The two parts of the likelihood function describe the relative and absolute possibility of observing the data in this case.

\textbf{Example} (Censoring) Suppose you have the survival data and assume the lifetime to follow exponential $\lambda$. Furthermore right censoring happens independently. Derive the MLE in this case.

Suppose the data is described as $(X_i,C_i)$, where $C_i = 1$ when censoring happens, and $0$ when no censoring happens. Then the 'likelihood' can be written as
$$
L(\lambda) = \prod_{i=1}^n (\lambda e^{-\lambda x_i} )^{1-c_i} (e^{-\lambda x_i} )^{c_i}
$$

\section{Conditional Expectations}

This is a topic which differs most from basic understanding to formal probability setting. When we consider the discrete case, it is well understood that $\P (Y | X =x)$ is well defined by the conditional probability formula. But the conditional probability formula can not properly define the conditional probability or the conditional expectation when $X$ is continuous.

Then why conditional expectation are of the special interest? The conditional expectation $\E (Y|X)$ is \textbf{a functional of $X$}. Actually it is the best $L^2$ approximation of $Y$ in all the measurable function of $X$.(You may find this sentence making no sense but it is OK now)

Then we have two property of the conditional expectations
$$
\E (Y) = \E (\E(Y|X))
$$
and
$$
\var{X} = \E (\var{X|Y}) + \var{\E (X|Y) }
$$

By using this we can show some interesting theorems.



% --------------------------------------------------------------
% You don't have to mess with anything below this line.
% --------------------------------------------------------------
\end{document} 