\documentclass[12pt]{article}
\usepackage{amssymb}
\usepackage{amsfonts}
\usepackage{amsfonts}
\usepackage{amssymb}
\usepackage{bbm}
\usepackage[margin=1in]{geometry}
\usepackage{amsmath,amsthm,amssymb}
\newcommand{\N}{\mathbb{N}}
\newcommand{\Z}{\mathbb{Z}}
\newcommand{\1}{\mathbbm{1}}
\newenvironment{theorem}[2][Theorem]{\begin{trivlist}
\item[\hskip \labelsep {\bfseries #1}\hskip \labelsep {\bfseries #2.}]}{\end{trivlist}}
\newenvironment{lemma}[2][Lemma]{\begin{trivlist}
\item[\hskip \labelsep {\bfseries #1}\hskip \labelsep {\bfseries #2.}]}{\end{trivlist}}
\newenvironment{exercise}[2][Exercise]{\begin{trivlist}
\item[\hskip \labelsep {\bfseries #1}\hskip \labelsep {\bfseries #2.}]}{\end{trivlist}}
\newenvironment{problem}[2][Problem]{\begin{trivlist}
\item[\hskip \labelsep {\bfseries #1}\hskip \labelsep {\bfseries #2.}]}{\end{trivlist}}
\newenvironment{question}[2][Question]{\begin{trivlist}
\item[\hskip \labelsep {\bfseries #1}\hskip \labelsep {\bfseries #2.}]}{\end{trivlist}}
\newenvironment{corollary}[2][Corollary]{\begin{trivlist}
\item[\hskip \labelsep {\bfseries #1}\hskip \labelsep {\bfseries #2.}]}{\end{trivlist}}
\newcommand{\E}{\operatorname{\mathbb{E}}}
\renewcommand{\P}{\operatorname{\mathbb{P}}}
\newcommand{\Var}{\operatorname{Var}}
\newcommand{\Cov}{\operatorname{Cov}}
\newcommand{\Cor}{\operatorname{Cor}}
\newcommand{\expect}[1]{\mathbb{E}\left(#1\right)}
\newcommand{\pr}[1]{\mathbb{P}\left(#1\right)}
\newcommand{\var}[1]{\operatorname{Var}\left(#1\right)}
\newcommand{\cov}[1]{\operatorname{Cov}\left(#1\right)}
%\newcommand{\cor}[1]{\operatorname{Cor}\left(#1\right)}
\newcommand\indep{\protect\mathpalette{\protect\independenT}{\perp}}
\def\independenT#1#2{\mathrel{\rlap{$#1#2$}\mkern2mu{#1#2}}}

\def\iid{\stackrel{\rm iid}{\sim}}
\def\Bin{\text{Bin}}
\def\Unif{\text{Unif}}
\def\lsto{\stackrel{\rm sto}{\leq}}
\def\gsto{\stackrel{\rm sto}{\geq}}

\begin{document}
% --------------------------------------------------------------
% Start here
% --------------------------------------------------------------
\title{4. Cremer, Rao, and Fisher}%replace X with the appropriate number
\author{GULP Winter 2015} %if necessary, replace with your course title
\maketitle

\section{Lower bound for the variance of unbiased estimators}

The discussion is under the assumption of squared loss. Therefore we have the bias-variance decomposition discussed last time. If we have an unbiased estimator, then the loss is equivalent to the variance of the estimator. Here we propose a lower bound of the variance of the unbiased estimators, which is independent of the form of the estimators.

Suppose we have a random sample $X$ and some unbiased estimator $\delta(X)$ for $g(\theta)$. Define a \textbf{score variable} $V = \frac{\partial}{\partial \theta} \ln f(X;\theta)$. We can derive several properties for this variable. Under some regular conditions (which enables us to switch the integration and the differential operator), 
$$
\E (V) = \int  [\frac{\partial}{\partial \theta} \ln f(x;\theta)] f(x;\theta) dx = \int \frac{1}{f(x;\theta)}[\frac{\partial}{\partial \theta} f(x;\theta)] f(x;\theta) dx = \int \frac{\partial}{\partial \theta} f(x;\theta)  dx = 0
$$
\begin{eqnarray*}
\cov{V,\delta}& =& \E (V\delta(X) ) = \int \delta(x) \frac{1}{f(x;\theta)}[\frac{\partial}{\partial \theta} f(x;\theta)] f(x;\theta) dx\\& =&  \int \frac{\partial}{\partial \theta} \delta(x) f(x;\theta)  dx =  \frac{\partial}{\partial \theta} \int \delta(x) f(x;\theta)  dx = g'(\theta)
\end{eqnarray*}

Using Cauchy-Schwartz inequality we have
$$
\sqrt{\var{V} \var{\delta}} \geq \cov{V,\delta},
$$
which is just
$$
\var{\delta} = \frac{\cov{V,\delta}^2}{\var{V}} = \frac{(g'(\theta))^2}{\E (V^2)}
$$

We define the right hand side as the \textbf{Cremer-Rao Bound}, and $\E (V^2) $ as the \textbf{Fisher information}, denoted as $I(\theta)$. By definition, 
$$
I(\theta) = \E ((\frac{\partial}{\partial \theta} \ln f(X,\theta))^2)
$$

\section{Some properties of the Fisher information}
If you have $n$ iid samples, your Fisher information will be $n$ times of the information carried by a single observation. It is necessary to have iid setting here. 

Another important equation which simplifies a lot of calculations is that 
$$
I(\theta) =  - \E((\frac{\partial^2}{\partial \theta^2} \ln f(X,\theta)))
$$
A sketch of proof is as follows. From the original definition we have 
\begin{eqnarray*}
I(\theta) & = & \E ((\frac{\partial}{\partial \theta} \ln f(X,\theta))^2) \\
& = & \int (\frac{1}{f(x;\theta)})^2[\frac{\partial}{\partial \theta} f(x;\theta)]^2 f(x;\theta) dx ,
\end{eqnarray*}
while 
\begin{eqnarray*}
 - \E((\frac{\partial^2}{\partial \theta^2} \ln f(X,\theta))) & = & -\int  (\frac{\partial^2}{\partial \theta^2} \ln f(x,\theta)) f(x,\theta) dx \\& = &  -\int 
\frac{\partial}{\partial \theta} (\frac{1}{f(x;\theta)} \cdot \frac{\partial}{\partial \theta} f(x;\theta) )  f(x;\theta) dx \\ & = & - \int\{ (\frac{1}{f(x;\theta)})^2[\frac{\partial}{\partial \theta} f(x;\theta)]^2 + \frac{1}{f(x;\theta)} \frac{\partial^2}{\partial \theta^2} f(x;\theta)\} f(x;\theta) dx \\ &= &  \int (\frac{1}{f(x;\theta)})^2[\frac{\partial}{\partial \theta} f(x;\theta)]^2 f(x;\theta) dx .
\end{eqnarray*}
For the latter part, just move the second order differentiate operator out of the integration sign and you will get zero.
% --------------------------------------------------------------
% You don't have to mess with anything below this line.
% --------------------------------------------------------------
\end{document} 